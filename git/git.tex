\documentclass{article}
\title{\vspace{-2.25cm}{\small Git}\vspace{-1.5cm}}
\date{}
\pagenumbering{gobble}
\usepackage[margin=1cm,landscape,a4paper]{geometry}
\usepackage{multicol}
\setlength{\columnseprule}{0.2pt}

\usepackage{listings}
\lstset{
	basicstyle=\ttfamily,
	columns=flexible,
	breaklines=true
}

\begin{document}
\maketitle
\begin{multicols}{3}
	
\section{Notes}
\begin{itemize}
\item You do not necessarily need to track every file when committing

\end{itemize}

\section{Data}
\begin{itemize}
\item some file is \texttt{alpha}
\end{itemize}

\section{Setup}
\subsection{Commands you will want to run on a fresh install}
\lstinline|$ git config --global user.email "user@example.com"|
\lstinline|git config --global user.name "John Smith"|

\subsection{Initialize Git}
\lstinline|$ git init|
\section{Commands}

\subsection{Amend local git commit}
\lstinline|$ git commit --amend|

\subsection{Combine last X commits}
\lstinline|$ git reset --soft HEAD~3 && git commit|

\subsection{Reset git Directory}
git reset --hard

\subsection{List Branches}
\lstinline|$ git branch|

\subsection{Git Stash}
\subsubsection{Store Changes}
\lstinline|$ git stash|

\subsubsection{List Stashes}
\lstinline|$ git stash list|

\subsection{Create Remote Branch}
\lstinline|$ git push --set-upstream origin branchname|

\section{Staging}
\subsection{Stage file}
\lstinline|$ git add alpha|

\subsection{Stage all files}
\lstinline|$ git add -A|
\lstinline|$ git add .|

\subsection{Stage all modified files}
\lstinline|$ git add -u|

\section{"Squashing"}
SHA should be the node of the branch to be squashed
\lstinline|$ git rebase -i [SHA]|
"pick" one commit, "f" the others
\lstinline|$ git commit --amend -e|
then edit commit message
\end{multicols}
\end{document}